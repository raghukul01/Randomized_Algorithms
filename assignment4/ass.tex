
%=======================   Default Templete   ==================
\documentclass[a4paper]{article}


% file with some default definations
\input{structure.tex}
\usepackage{listings}
\lstset{language=Python, basicstyle=\normalsize\sffamily\linespread{0.8}, numbers=left, numberstyle=\small, stepnumber=1, numbersep=5pt}
\usepackage{fancyhdr}
\setlength{\parindent}{0pt}

\pagestyle{fancy}
\fancyhf{}
\lhead{\textbf{\NAME\ (\ANDREWID)}}
\chead{\textbf{Assignment \HWNUM}}
\rhead{\COURSE}


%==================Header details======================
\newcommand\NAME{Raghukul, Vibhor}
\newcommand\ANDREWID{160538, 160778}
\newcommand\HWNUM{4}
\newcommand\COURSE{CS648}
%======================================================

% available formatted sections:
% - COMMAND LINE ENVIRONMENT: \begin{commandline} \end{commandline}
% - FILE CONTENTS ENVIRONMENT: \begin{file}[optional filename, defaults to "File"]
% - NUMBERED QUESTIONS ENVIRONMENT: \begin{question}[optional title]
% - WARNING TEXT ENVIRONMENT(can also be used for note): \begin{warn}[optional title, defaults to "Warning:"]
% - INFORMATION ENVIRONMENT(can be used to mention given details): \begin{info}[optional title, defaults to "Info:"]

%===============================================================
\begin{document}

% start of Q1
\begin{question}
\textbf{Internalizing Min-cut Algorithm}
\end{question}
\subsection*{1.}

\subsection*{2.}
\textbf{Time Complexity :}\\
Each contract step takes $O(n)$ and we are contracting $\frac{n}{2}$ times so this results in $O(n^2)$ work.\\
Thus if the time complexity of the algorithm is $T(n)$ then,
$$T(n) = O(n^2) + 2T(\frac{n}{2})$$
$$T(n) = O(n^2) + O(\frac{n^2}{2}) + 2T(\frac{n}{4})$$
$$T(n) = O(n^2) + O(\frac{n^2}{2}) + O(\frac{n^2}{8}) + \dots$$
$$T(n) = O(n^2)$$
\textbf{Error Analysis :}\\
Let the probability that algorithm preserves the min-cut be $q(n)$.\\
As shown in class, probability that min-cut is preserved after $\frac{n}{2}$ contractions is at least $\frac{1}{4}$.
Thus, we can say that $q(n)$ is at least $\frac{1}{4}$ times the probability that at least one smaller recursive call of size $\frac{n}{2}$ preserves the min-cut.
So,
$$q(n) \geq \frac{1}{4}\left(1-\left(1-q\left(\frac{n}{2}\right)\right)^2\right)$$
$$q(n) \geq \frac{1}{4}\left(2 - q\left(\frac{n}{2}\right)\right)q\left(\frac{n}{2}\right)$$
$$q(n) \geq \frac{1}{2}q\left(\frac{n}{2}\right) - \frac{1}{4}q\left(\frac{n}{2}\right)^2$$
Now let $p(k) = q(2^k)$. So we have,
$$p(k) \geq \frac{1}{2}p\left(k-1\right) - \frac{1}{4}p\left(k-1\right)^2$$
Call the above inequality to be $I_1$.\\
Since $p(k) \leq 1\ \forall k$ we get $$p(k) \geq \frac{1}{2}p\left(k-1\right) - \frac{1}{4}p\left(k-1\right)$$
or,
$$p(k) \geq \frac{1}{4}p\left(k-1\right)$$
By the above inequality and $I_1$ we have,
$$p(k) \geq \frac{1}{2}p\left(k-1\right) - p(k)p\left(k-1\right)$$
Thus,
$$\frac{1}{p(k)} \leq \frac{2}{p(k-1)} + 2$$
Base case - $p(1) = 1$
$$\frac{1}{p(k)} \leq 2\left(\frac{2}{p(k-2)} + 2\right) + 2$$
$$\frac{1}{p(k)} \leq \frac{2^2}{p(k-2)} + 2 + 2^2$$
In general,
$$\frac{1}{p(k)} \leq \frac{2^i}{p(k-i)} + 2 + 2^2 + \dots + 2^i$$
So,
$$\frac{1}{p(k)} \leq \frac{2^{k-1}}{p(1)} + 2 + 2^2 + \dots + 2^{k-1}$$
$$\frac{1}{p(k)} \leq 2^{k-1} + 2(2^{k-1}-1)$$
$$\frac{1}{p(k)} \leq 2^k + 2^{k-1} - 2$$
Thus,
$$\frac{1}{q(n)} \leq n + \frac{n}{2} - 2$$
$$q(n) \geq \frac{2}{3n - 4}$$
So the probability that the algorithm preserves min-cut is at least $\frac{2}{3n - 4}$\\
To get error probability less than $\frac{1}{n^c}$ repeat our algorithm $cn\log(n)$ times.\\
Hence, the time complexity of the complete algorithm to get an inverse polynomial error bound is $O(n^3 \log(n))$.
\pagebreak
% start of Q2
\begin{question}[]
\textbf{A surprising problem from computational geometry}
\end{question}
\subsubsection*{1.}
We will construct the line segments in order they are to be selected in the permutation. Let's say the first line segment is a horizontal line segment of length 1 having its' left end at origin. Construct the second line segment such that it will intersect the extended first line segment on the right side, and makes a small positive angle with the horizontal. 
Construct the third line segment which will intersect both the previous extended line segments, this line segment makes just greater angle with the horizontal than the previous line segment.
Now consider that we have constructed $i-1$ line segments in this manner, now construct the $i^{th}$ line segment intersecting all previous line segments and making an angle just greater than the ${(i-1)}^{th}$ line segment with the horizontal. Do this till we get $n$ line segments.\\
\textbf{Note - }I am considering angle with the horizontal in the anticlockwise sense.\\
This procedure will result in $n\choose2$ points of intersection as every pair of line segments will have a distinct point of intersection if constructed this way.\\
Thus, there exist a set of $n$ segments and a permutation for which the above algorithm will cause $n\choose2$ points of intersection.
\subsubsection*{2.}

\pagebreak

% start of Q3
\begin{question}[]
\textbf{Internalizing Backward Analysis and Randomized Incremental
Construction (RIC)}
\end{question}

\end{document}

