
%=======================   Default Templete   ==================
\documentclass[a4paper]{article}


% file with some default definations
\input{structure.tex}
\usepackage{listings}
\lstset{language=Python, basicstyle=\normalsize\sffamily\linespread{0.8}, numbers=left, numberstyle=\small, stepnumber=1, numbersep=5pt}
\usepackage{fancyhdr}
\setlength{\parindent}{0pt}

\pagestyle{fancy}
\fancyhf{}
\lhead{\textbf{\NAME\ (\ANDREWID)}}
\chead{\textbf{Assignment \HWNUM}}
\rhead{\COURSE}


%==================Header details======================
\newcommand\NAME{Raghukul, Vibhor}
\newcommand\ANDREWID{160538, 160778}
\newcommand\HWNUM{2}
\newcommand\COURSE{CS648}
%======================================================

% available formatted sections:
% - COMMAND LINE ENVIRONMENT: \begin{commandline} \end{commandline}
% - FILE CONTENTS ENVIRONMENT: \begin{file}[optional filename, defaults to "File"]
% - NUMBERED QUESTIONS ENVIRONMENT: \begin{question}[optional title]
% - WARNING TEXT ENVIRONMENT(can also be used for note): \begin{warn}[optional title, defaults to "Warning:"]
% - INFORMATION ENVIRONMENT(can be used to mention given details): \begin{info}[optional title, defaults to "Info:"]

%===============================================================
\begin{document}

% start of Q1
\begin{question}
\textbf{Using prime numbers for 2-D pattern matching}
\end{question}
Some Results that we know are:
\begin{itemize}
    \item{$M(\varepsilon) = \begin{bmatrix} 1 & 0 \\ 0 & 1 \end{bmatrix} $ \hfill \textit{(i)}}
    \item{$M(0) = \begin{bmatrix} 1 & 0 \\ 1 & 1 \end{bmatrix} $ \hfill \textit{(ii)}}
    \item{$M(1) = \begin{bmatrix} 1 & 1 \\ 0 & 1 \end{bmatrix} $ \hfill \textit{(iii)}}
    \item{For non-empty strings $x$ and $y$, $M(xy) = M(x) × M(y)$
    \hfill \textit{(iv)}
    }
    \item{$M(x)$ is defined for all $x\in \{0,1\}^{*}$ \hfill \textit{(v)}}
    \item{$M(x) = M(y) \Rightarrow x = y$ \hfill \textit{(vi)}}
\end{itemize}
\subsection*{(a)}
Note that $M(z)$ a string $z$ of length $k$ can be calculated recursively as: \\
\noindent\rule{\textwidth}{0.4pt}
\textbf{Algorithm 1:} \\
\noindent\rule{\textwidth}{0.2pt}
\begin{lstlisting}[columns=fullflexible] 
def compute_M(z, k):    
    if(k = 1)
        if(z[0] = 0)
            return M(0)
        else
            return M(1)
    else
        return M(z[0])*compute_M(z+1, k-1)
\end{lstlisting}
% \noindent\rule{\textwidth}{0.4pt}
\vspace{1.5em}
Time Complexity of above algorithm $= O(k)$. So now we have a method to compute fingerprint in form of $2 \times 2$ matrix, in $O(k)$ where $k$ is length of string. First we compute figerprint of pattern in $O(m)$ time, and the we compute fingerprint for all continuous strings of length $m$ ($n - m + 1$ such strings) in text. and then we comapare these $n - m + 1$ fingerprints with fingerprint of pattern. By result \textit{$(vi)$} we know that if fingerprint are equal, then so is the actual string. Now we need to find an efficient way to compute fingerprint of all the strings of length $m$ in text.

\subsubsection*{Computation of M for all string in text of length \boldmath$ m $}
let text ($t$) = $t_1t_2\ldots.t_n$, $M_{i}$ be the fingerprint($M$) of $t[i,m + i - 1]$. Now how can we compute $M_{i+1}$ efficiently from $M_i$? Note that $M_{i+1}$ is fingerprint of $t[i+1,m+i] = t_{i+1} t_{i+2}\ldots t_{m+i}$, and $t[i,m +i-1] = t_i t_{i+1}\ldots t_{m+i-1}$.
$$M_{i+1} = M(t_{i+1})M(t_{i+2})\ldots M(t_{m+i})$$
$$M_1 = M(t_{i})M(t_{i+1})\ldots M(t_{m+i-1})$$
hence we get:
$$M_{i+1} = M(t_i)^{-1} * M_i * M(t_{m+i}) $$
Since any operation involving aithematics is assumed to be in $O(1)$, the above computation can be done in constant time, with memoization.When we have the fingerprint for all $M_i$, we just compare them with fingerprint of pattern, and report if we find a collision.

\pagebreak

\noindent\rule{\textwidth}{0.4pt}
\textbf{Algorithm 2:} \\
\noindent\rule{\textwidth}{0.2pt}
\begin{lstlisting}[columns=fullflexible] 
T := text of length n
P := pattern of length m
M[1:n-m+1] := list to store fingerprints of T
N[0:1] := list to store fingerprint of base case

N[0] = {1, 0, 1, 1}       # base case
N[1] = {1, 1, 0, 1}       # base case

M_P := compute_M(P, m)
M[1] = compute_M(T, m)

for i in (2,n-m+1)
    M[i] = matrix_inverse(N[ T[ i - 1 ] ]) * M[i - 1] * N[ T[ m + i - 1 ] ]

for i in (1,n-m+1)
    if (M[i] = M_P)
        return i
\end{lstlisting}
\vspace{1.5em}
\subsubsection*{Time Complexity:}
Note that compute\_M takes $O(k)$ time for a string of length $k$. So computation of M\_P would take $O(m)$ time, similar to this computation of M[1] would also take $O(m)$ time. Since computation of M[i] from M[i - 1] takes $O(1)$ time, time complexity of computing M[] is $O(n-m)$, and in the end we just do a $O(1)$ comparison $n-m+1$ time.
$$T_{n,m} = O(m) + O(m) + O(n-m) + O(n-m + 1)$$
$$T_{n,m} = O(2n + 1) \approx O(n + m) \text{(since m < n)}$$

\subsection*{(b)}
Let $M(P)$ be the fingerprint of pattern.
The only case when the algorithm in $(b)$ will give wrong answer, is when the pattern and string don't match, but $M_i = M(P)$. Or $M_i - M(P) = 0 \mod p$, which means that all the $4$ entries of the matrix $M_i - M(P) = M(E)$ are zero modulo $p$.\\
So the event when error will occur is defined as: $M(E)_{11}, M(E)_{12}, M(E)_{21}, M(E)_{22} = 0 \mod p$ and P, T[i:i+m-1] don't match. \\
Let $P(\delta)$ denote the error probability (note that below equalities are given $\mod p$)
$$P(\delta) = P(M(E)_{11} = 0 \cap M(E)_{12} = 0 \cap M(E)_{21} = 0 M(E)_{22} = 0)$$
$$P(\delta) \leq P(M(E)_{11} = 0) \text{ since $\delta$ is a subset of this event}$$

Let $\pi(x)$ denotes the no of primes less than or equal to $ x$. \\
Let $X(a)$ be the no of primes which divide $a$. \\
Clearly, $$a \geq 2^{X(a)}$$
Thus, $$X(a) \leq \log(a)$$
\begin{warn}[Lemma 1: Probability that $A\ mod\ p = 0$ when $p$ is a random prime no from $(2,t)$ is less than or equal to $\frac{log(A}{\pi(t)}$.]
\end{warn}

\begin{proof}
Since $p$ is selected randomly uniformly from all primes from $(2,t)$,
$$P(A\ mod\ p = 0) = \frac{X(A)}{\pi(t)}$$
Thus, 
$$P(A\ mod\ p = 0) = \frac{\log (A)}{\pi(t)}$$
\end{proof}
\begin{warn}[Note: Here we are only considering the first instance where the algorithm fails for our analysis]
\end{warn}
Note that $F_n < 2^n$ where $F_n$ is $n^{th}$ fibonacci number, also the terms of $M(Z)$ are bounded by $F_n$ (given in problem).\\
we know: $$M(Z)_{11} < 2^n$$
By above lemma we know that: $$P(\delta) \leq \dfrac{\log (2^n)}{\pi(t)}$$
$$P(\delta) \leq \frac{n}{\pi(t)}$$
Now,
$$\pi(t) \approx \frac{t}{log(t)}$$
Thus, $$P(\delta) \leq \frac{n*log(t)}{t}$$

Take $t > 5*n^5log(n)$. Then, $$P(\delta) \leq n^{-4}$$
Thus for $t > 5*n^5log(n)$, probability that fingerprint will match, provided actual string don't match, would be less that $n^{-4}$.

\subsection*{(c)}
In this problem we use the fingerprinting approach, we compute fingerprint of matrices of size $m \times m$, same way as we did above.First we flatten the 2D matrix into a 1D matrix of size $m^2$, by appending $2^{nd}$ row after $1^{st}$ row, and so on. This computation for each matrix in text of size $m \times m$, needs to be computed efficienty, to achieve $O(n^2)$ Complexity. \\ 
Note that here also our fingerprint is a 2D matrix of size $2 \times 2$. \\
\noindent\rule{\textwidth}{0.4pt}
\textbf{Algorithm 1:} \\
\noindent\rule{\textwidth}{0.2pt}
\begin{lstlisting}[columns=fullflexible] 
T := n x n matrix in which we need to search for pattern
P := pattern (dimension m x m)
p := prime number choosen uniformly and randomly from (2, t)
N[0:1] := list to store fingerprint of base case

N[0] = {1, 0, 1, 1}       # base case
N[1] = {1, 1, 0, 1}       # base case

# first we compute the fingerprint of all continuous strings of length m in
# each row (n-m+1 such string in each row)
F := 2D list to store above fingerprint
# F[i][j] stores fingerprint of string in ith row starting at index j

for i in (1,n)
    F[i][1] = compute_M(T[i], m) mod p
for i in (1,n)
    for j in (2,n-m+1)
        F[i][j] = matrix_inverse(N[T[i][j - 1]]) * F[i][j - 1] * N[T[i][m + j - 1]] mod p

# store fingerprint of pattern
P_new = flatted(P)
M_P = compute_M(P_new, m*m) mod p

# Now we compute fingerprint of all continuous matrices of size m x m using F[i][j]
G := 2D list to store this fingerprint

# G[i][j] denote fingerprint of matrix with left top at i,j and right bottom at i+m-1, j+m-1
# Since fingerprint is calculated by flattening, hence G[i][j] = F[i][j]*F[i+1][j]*...F[i+m-1][j]

# we do a column wise traversal
for j in (1, n-m+1)
    G[1][j] = F[1][j]*F[2][j]*...F[m][j] mod p
for j in (1, n-m+1)
    for i in (2, n-m+1)
        G[i][j] = matrix_inverse(F[i - 1][j]) * G[i - 1][j] * F[i+m-1][j] mod p

# fingerprint matching
for i in (1, n-m+1)
    for j in (1, n-m+1)
        if (G[i][j] = M_P)
            return (i,j)
\end{lstlisting}
\vspace{1.3em}

\subsubsection*{Time Complexity}
Computing matrx F[i][1] for all $i \in (1, n)$ would take $O(m) * n$ time. Computing rest of the entries of F would take $O(1)$ time per entry, so a total of $O(n*(n-m))$. Calculation of M for P would take $O(m^2)$ time. Computing G[1][j] for all $j\in (1,n-m+1)$ would take $O(m) * (n-m+1)$ time. And lastly computation of remaining entries of G would take $O(1)$, for each entry, so total $O((n-m)*(n -m + 1))$. So in total we get 
$$T_{n,m} = O(mn) + O(n(n - m)) + O(m^2) + O(m*(n - m + 1)) + O((n-m)*(n-m+1))$$
Or, $$T_{n,m} = O(2n^2 + m^2 - mn + n - 2m) \approx O(n^2) \text{ since m < n}$$


\subsubsection*{Error Probability}
Error analysis is similar to previous problem, the only thing that changes is that now $M[i][j]_{11}$ would be bounded by $F_{n^2}$, since we are flattening 2D square matrix.\\
Note that $F_n < 2^n$ where $F_n$ is $n^{th}$ finbonacci number, also the terms of $M[i][j]$ are bounded by $F_{n^2}$.\\
we know: $$M(Z)_{11} < 2^{n^2}$$
By above lemma we know that: $$P(\delta) \leq \dfrac{\log (2^{n^2})}{\pi(t)}$$
$$P(\delta) \leq \frac{n^2}{\pi(t)}$$
Now,
$$\pi(t) \approx \frac{t}{log(t)}$$
Thus, $$P(\delta) \leq \frac{n^2*log(t)}{t}$$

Take $t > 10*n^6log(n)$. Then, $$P(\delta) \leq n^{-4}$$
Thus for $t > 10*n^6log(n)$, probability that fingerprint will match, provided actual matrices don't match, would be less that $n^{-4}$.

\pagebreak
% start of Q2
\begin{question}[]
\textbf{(1) Let $a_k$ denote the element of $S$ inserted during $k^{th}$ iteration in the above algorithm.
Let $\varepsilon_k$ be the event that $a_k$ is inserted in table $T_2$. Calculate $P(\varepsilon_k)$.\\
(2) Let $X_k$ denote the random variable defined as follows. If $a_k$ is stored in $T_1$, then
$X_k = 0$; otherwise, it is the number of elements among $\{a_1, . . . , a_{k-1}\}$ that collide with it in $T_2$.
Provide a suitable upper bound on $E[X_k | E_k]$.\\
(3)  Use (2) to show that we can design a perfect hashing (at most one element in any
location of $T_1$ and $T_2$) using $n$ which is asymptotically much smaller than $s^2$.}
\end{question}
\subsubsection*{(1)}
    Probability of event $\varepsilon_k$ would be the same as the probability of the event that $a_k$ collides with at least one of $a_1, a_2, ... , a_{k-1}$ under hash function $h_1$. Call this event to be $E$.
    
    Call the event that $a_k$ collides with $a_i$ to be $E_i$ where $1\leq i\leq (k-1)$
    Now, $$P(E) = P(\bigcup_{i=1}^{k-1} E_i)$$
    Since $h_1$ is selected randomly uniformly from universal hash family $H$ we get for any $a,b\ \epsilon\ U$,
    $$P(h_1(a) = h_1(b)) \leq \frac{c}{n}$$
    Thus,
    $$P(E_i) \leq \frac{c}{n}$$
    By union theorem,
    $$P(E) \leq P(\sum_{i=1}^{k-1} E_i)$$
    $$P(E) \leq \frac{(k-1)c}{n}$$
    also we have $P(\varepsilon_k) = P(E)$.
    Hence we get,
    $$P(\varepsilon_k) \leq \frac{(k-1)c}{n}$$

\subsubsection*{(2)}
Given that $X_k$ denote the random variable defined as follows:
\begin{itemize}
    \item{If $a_k$ is stored in $T_1$, then $X_k = 0$}
    \item{else, it is the number of elements among $\{a_1,\ldots a_{k-1} \}$}, that collide with it in $T_2$.
\end{itemize}
Let us assume that $x$ elements are present in $T_2$ just before inserting $a_k$. Lets calculate the expected number of collision that $a_k$ will have in $T_2$. Let $Y_i$ ($i \in (1,x)$) be a random variable defined as follows:

\begin{align*}
Y_i & = \begin{cases} 
        1 \quad \text{if element $z_i$ collides with $x_k$ in $T_2$} \\
        0 \quad \text{otherwise}
\end{cases}
\end{align*}
let $Y = \sum_{i=1}^{x}Y_i$, then we need to calculate $E[Y]$. By the defination of perfect hashing function we know that:
$$P(h_2(a) = h_2(b)) \leq \frac{c}{n}$$, so now by applying linearity of expectation we get that $E[Y] \leq \dfrac{xc}{n}$.\\
So we can say that $E[X_k |\xi_{k}] \leq \dfrac{E[x]c}{n}$. Now problem reduces to calculate $E[x]$, which can be done using previous result. Let $Z_i$ ($i \in (1,k-1)$) be a random variable defined as follows:
\begin{align*}
Z_i & = \begin{cases} 
        1 \quad \text{if element $x_i$ moves to $T_2$} \\
        0 \quad \text{otherwise}
\end{cases}
\end{align*}
$Z_1$ is random variable corresponding to first element, and similarly for other values. So by linearity of expectation, we get $E[X] \leq \sum_{i=1}^{k-1} \dfrac{(i-1)c}{n}$ (using previous part of problem). 
$$E[X_k |\xi_{k}] \leq \dfrac{E[x]c}{n}$$
$$E[X_k |\xi_{k}] \leq \dfrac{c}{n}\sum_{i=1}^{k-1} \dfrac{(i-1)c}{n}$$
$$E[X_k |\xi_{k}] \leq \dfrac{c^2(k-1)(k-2)}{2n^2}$$

\subsubsection*{(3)}
Let the random variable denoting the total number of collisions is $X$.

Since collisions are contributed only by table $T_2$,
$$X \leq X_1 + ... + X_s$$
So trivially,
$$E[X] \leq \sum_{k=1}^{s}P(\varepsilon_k)E[X_k|\varepsilon_k]$$
$$E[X] \leq \sum_{k=1}^{s}\frac{(k-1)c}{n}\dfrac{c^2(k-1)(k-2)}{2n^2}$$
$$E[X] \leq \sum_{k=1}^{s}C\frac{k^3}{n^3}$$
$$E[X] \leq C\left(\frac{s(s+1)}{2}\right)^2\frac{1}{n^3}$$
$$E[X] \leq R\frac{(s+1)^4}{n^3}$$
To get expected no of collisions less than $1/2$ take $n = m(s+1)^{\frac{4}{3}}$ where $m$ is a suitable constant.

So we have $E[X] \leq \frac{1}{2}$ when $n = O(s^{\frac{4}{3}})$

Probability of no collision, $$P(X = 0) = 1 - P(X \geq 1)$$
Using Markov's inequality,
$$P(X = 0) \geq 1 - \frac{1}{2}$$
$$P(X = 0) \geq \frac{1}{2}$$

\pagebreak
\noindent\rule{\textwidth}{0.4pt}
\textbf{Algorithm 1:} \\
\noindent\rule{\textwidth}{0.2pt}
\begin{lstlisting}[columns=fullflexible] 
repeat
    Pick h2 randomly unniformly from H
    # The next step takes O(s) time
    t = number of collisions for S using the algorithm given
until t=0
for each a in S
    i = h1(a)
    if T1(i) is empty then
        Insert a in the list T(i)
    else
        j = h2(a)
        Insert a in list T2(j)
\end{lstlisting}
\vspace{1.3em}

Thus the perfect hash function can be calculated using the above algorithm in $O(s^{\frac{4}{3}})$ space.
\pagebreak



% start of Q3
\begin{question}[]
\textbf{Consider a collection $X_1, ..., X_n$ of $n$ independent geometrically distributed random variables with expected value $2$. Let $X = \sum_{i=1}^NX_i$ and $\delta > 0$\\
(1) Derive a bound on $P(X \geq (1 + \delta)(2n))$ by appyling the Chernoff bound to a sequence of $(1 + \delta)(2n)$ fair coin tosses.} \\
\textbf{(2) Directly derive a Chernoff like bound on $P(X \geq (1 + \delta)(2n))$ from scratch.} \\
\textbf{(3) Which bound is better?}

\end{question}
Let $Y$ be a geometrically distributed random variable then $P(Y=k) = (1-p)^{k-1}p$ and $E[Y]=\frac{1}{p}$.
\subsubsection*{(1)}
Since $E[X_i] = 2$ we find that each $X_i$ is geometrically distributed with parameter $p=\frac{1}{2}$.

As $p=\frac{1}{2}$ we can think of each $X_i$ as the number of coin tosses required to get the first head.

So their sum $X = \sum_{i=1}^nX_i$ can be thought of as the number of coin tosses required to get $n$ heads.

We have to find the probability that no less than $(1+\delta)(2n)$ coin tosses are required to get $n$ heads, this is same as the probability that less than $n$ heads appear in $(1+\delta)(2n)$ coin tosses which in turn is same as the probability that no less than $n(1+2\delta)$ tails appear in $(1+\delta)(2n)$ coin tosses

Let $Y_i$ be the random variable taking value $1$ when $i^{th}$ coin toss results in a tail and $0$ otherwise. Each $Y_i$ is a Bernoulli random variable with probability equal to $\frac{1}{2}$

Consider $Y = \sum_{i=1}^{(1+\delta)(2n)}Y_i$. By linearity of expectation, $E[Y] = (1+\delta)n$

Now, $$P(X \geq (1+\delta)(2n)) = P(Y \geq n(1+2\delta))$$
$$P(Y \geq n(1+2\delta)) = P\left(Y \geq \frac{(1+\delta)n(1+2\delta)}{1+\delta}\right)$$
$$P(Y \geq n(1+2\delta)) = P\left(Y \geq (1+\delta)n\left(1+\frac{\delta}{1+\delta}\right)\right)$$
$$P(Y \geq n(1+2\delta)) = P\left(Y \geq E[Y]\left(1+\frac{\delta}{1+\delta}\right)\right)$$
Applying chernoff bound on $Y$,
$$P(Y \geq n(1+2\delta)) \leq \frac{e^{n\delta}(1+\delta)^{n(1+2\delta)}}{(1+2\delta)^{n(1+2\delta)}}$$
$$P(Y \geq n(1+2\delta)) \leq \left(\frac{e^{\delta}(1+\delta)^{(1+2\delta)}}{(1+2\delta)^{(1+2\delta)}}\right)^n$$
Thus we get,
$$P(X \geq (1+\delta)(2n)) \leq \left(\frac{e^{\delta}(1+\delta)^{(1+2\delta)}}{(1+2\delta)^{(1+2\delta)}}\right)^n$$
\subsubsection*{(2)}
$$P(X \geq (1+\delta)(2n)) = P(e^{tX} \geq e^{t(1+\delta)(2n)})$$
By Markov's inequality,
$$P(e^{tX} \geq e^{t(1+\delta)(2n)}) \leq \frac{E[e^{tX}]}{e^{t(1+\delta)(2n)}}$$
Now,
$$E[e^{tX}] = E[e^{(tX_1 + ... + tX_n)}]$$
$$E[e^{tX}] = E[e^{tX_1}e^{tX_2}...e^{tX_n}]$$
$$E[e^{tX}] = E[\Pi_{i=1}^ne^{tX_i}]$$
We know that $X_i$s are independent so $e^{tX_i}$s are also independent amd hence we get,
$$E[e^{tX}] = \Pi_{i=1}^nE[e^{tX_i}]$$
and $$E[e^{tX_i}] = \sum_{j=1}^\infty e^{tj}P(X_i=j)$$
$$E[e^{tX_i}] = \sum_{j=1}^\infty e^{tj}(1-p)^{j-1}p$$
$$E[e^{tX_i}] = e^tp\sum_{j=1}^\infty {(e^t(1-p))}^{j-1}$$
We had $p=\frac{1}{2}$. Take $t < \ln 2$ so that $e^t(1-p) < 1$.

By using the formula of summation of a GP,
$$E[e^{tX_i}] = e^tp\frac{1}{1 - e^t(1-p)}$$
$$E[e^{tX}] = E[\Pi_{i=1}^ne^{tX_i}]$$
$$E[e^{tX}] = \left(\frac{e^tp}{1 - e^t(1-p)}\right)^n$$
Thus we get,
$$P(e^{tX} \geq e^{t(1+\delta)(2n)}) \leq \frac{1}{e^{t(1+\delta)(2n)}}\left(\frac{e^tp}{1 - e^t(1-p)}\right)^n$$
$$P(e^{tX} \geq e^{t(1+\delta)(2n)}) \leq \left(\frac{e^tp}{(1 - e^t(1-p))(e^{2t(1+\delta)})}\right)^n$$
This is true for every value of $t < \ln 2$ and hence differentiating it to find the minimum value - 
$$\frac{d}{dt}\left(\frac{e^tp}{(1 - e^t(1-p))(e^{2t(1+\delta)})}\right)^n = 0$$
$$\frac{d}{dt}\frac{e^tp}{(1 - e^t(1-p))(e^{2t(1+\delta)})} = 0$$
$$-(1-e^t(1-p))(1+2\delta)(e^{-t(1+2\delta)}) + (e^{-t(1+2\delta)})(e^t(1-p)) = 0$$
$$e^t(1-p) = (1+2\delta)(1-e^t(1-p))$$
$$e^t(1-p) = \frac{1+2\delta}{2+2\delta}$$
Put $p=\frac{1}{2}$,
$$e^t = \frac{1+2\delta}{1+\delta}$$
So we get after substituting this value of $e^t$,
$$P(e^{tX} \geq e^{t(1+\delta)(2n)}) \leq \left(\frac{(1+\delta)^{2+2\delta}}{(1+2\delta)^{1+2\delta}}\right)^n$$
So, $$P({X} \geq {(1+\delta)(2n)}) \leq \left(\frac{(1+\delta)^{2+2\delta}}{(1+2\delta)^{1+2\delta}}\right)^n$$

\subsubsection*{(3)}
Let the first and second bounds are $B1$ and $B2$ respectively,
$$\frac{B1}{B2} = \frac{e^\delta}{1+\delta}$$
$$\frac{B1}{B2} \geq 1$$
Thus, the second bound is better.

\end{document}

