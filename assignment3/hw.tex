
%=======================   Default Templete   ==================
\documentclass[a4paper]{article}


% file with some default definations
\input{structure.tex}
\usepackage{listings}
\lstset{language=Python, basicstyle=\normalsize\sffamily\linespread{0.8}, numbers=left, numberstyle=\small, stepnumber=1, numbersep=5pt}
\usepackage{fancyhdr}
\setlength{\parindent}{0pt}
\usepackage{algorithm2e}
\pagestyle{fancy}
\fancyhf{}
\lhead{\textbf{\NAME\ (\ANDREWID)}}
\chead{\textbf{Assignment \HWNUM}}
\rhead{\COURSE}


%==================Header details======================
\newcommand\NAME{Raghukul, Vibhor}
\newcommand\ANDREWID{160538, 160778}
\newcommand\HWNUM{3}
\newcommand\COURSE{CS648}
\newcommand{\V}[1]{\boldsymbol{#1}}
%======================================================

% available formatted sections:
% - COMMAND LINE ENVIRONMENT: \begin{commandline} \end{commandline}
% - FILE CONTENTS ENVIRONMENT: \begin{file}[optional filename, defaults to "File"]
% - NUMBERED QUESTIONS ENVIRONMENT: \begin{question}[optional title]
% - WARNING TEXT ENVIRONMENT(can also be used for note): \begin{warn}[optional title, defaults to "Warning:"]
% - INFORMATION ENVIRONMENT(can be used to mention given details): \begin{info}[optional title, defaults to "Info:"]

%===============================================================
\begin{document}

% start of Q1
\begin{question}
\textbf{Estimating all-pairs distances exactly}
\end{question}
\begin{algorithm}[H]
\SetAlgoLined
\KwIn{$G$, partial distance matrix $M$}
\KwOut{complete distance matrix $M$ with error probability $< 1 - 1/n^2$}
 $D \leftarrow n$ x $n$ matrix to store distance \\ 
 \For{$i \leftarrow 1$ \KwTo $n$} {
 	\For{$j \leftarrow 1$ \KwTo $n$} {
 		\uIf{$M[i][j] =$ \#}{
 			$D[i][j]$ = infinity
 		}
 		\uElse {
 			$D[i][j] = M[i][j]$
 		}
 	}
 }
 \For{$k \leftarrow 1$ \KwTo $K$} {
 	$w \leftarrow$ Random vertex in $[1:n]$ \\
 	$dist = bfs($G, w$)$ \\
 	\For{$i \leftarrow 1$ \KwTo $n$} {
 		\For{$j \leftarrow 1$ \KwTo $n$} {
 		    $D[i][j] = min(D[i][j], dist[i] + dist[j])$
 		}
 	}
 }

 \Return{$M$}
 \caption{Monte Carlo Algorithm for exact distance}
\end{algorithm}
\subsection*{Complexity analysis}
In above algorithm $bfs(G, s)$ returns an array storing shortest distance of $s$ to each vertex in $G$.
Time complexity for $bfs$ is $O(n + m)$ ($m$ = number of vertices in graph).
In worst case there can be atmost $n \choose 2$ edges, i.e worst case complexity of $bfs$ is $O(n^2)$. Time taken to update $D$ in one iteration is also $O(n^2)$.
Since outer loop runs for $K$ times, total complexity is $O(n^2K)$.
In below analysis it is show that $K = O(\log n)$, hence complexity of above algorithm is $\V{O(n^2)\log n}$.

\subsection*{Error Analysis}

Error will occur if there are $\geq 1$ entries in $D$ which are not equal shortest path distance.
\begin{align*}
P(error) &= P(M[1][1] \text{ wrong } \cup M[1][2] \text{ wrong } \ldots \cup M[n][n]\text{ wrong })
\end{align*}
Applying Boole's inequality we get:
\begin{align*}
P(error) &\leq n^2*P(M[i][j] \text{ wrong }) \text{ ( distance between $i,j > n/100$ )}
\end{align*}
Note that there is $\leq$ sign in above line since the points for which distance $\leq n/100$ matrix $M$ stores correct value (since the value is taken directly from partial distance matrix).

\subsubsection*{$\V{M[i][j]}$ computed wrong}
We know that distance between $i, j$ is greater than $n/100$, i.e number of nodes between shortest path from $i$ to $j$ is $\geq n/100$.
So now consider first iteration of first loop. \\
Note that $w$ is choosen randomly, hence the probability that $w$ belongs to shortest path from $i$ to $j$ is $\geq \dfrac{n/100}{n} = \dfrac{1}{100}$. \\
If $w$ belong to shortest path then $D[i][j]$ will be updated correctly and will contain shortest path distance (since we are doing $bfs$ from $w$, we get shortest path value from $w\rightarrow i, w\rightarrow j$, $M[i][j]$ is simple the sum of these $2$ values).\\
Probability that $D[i][j]$ is wrong ($P(\delta)$) $\leq \dfrac{99}{100}$ \\
Probability that $D[i][j]$ is wrong even after $K$ iteration $\leq P(\delta)^{K} = \Bigg( \dfrac{99}{100} \Bigg)^K$ \\
Using this value in above equation we get

\begin{align*}
P(error) &\leq n^2 P(\delta) \\
		 &\leq n^2 \Bigg( \dfrac{99}{100} \Bigg)^K
\end{align*}
We want that error probability is $\leq 1/n^2$, or
\begin{align*}
P(error) &\leq \dfrac{1}{n^2} \\
n^2 \Bigg( \dfrac{99}{100} \Bigg)^K &\leq \dfrac{1}{n^2} \\
\Bigg( \dfrac{99}{100} \Bigg)^K &\leq \dfrac{1}{n^4}\\
K\log \Bigg( \dfrac{100}{99}\Bigg) &\geq 4\log n \\
K &\geq \V{\dfrac{4}{\log(100/99)} \log n}
\end{align*}
Taking $\V{K = 400\log n}$ we get that all entries of the distance matrix are correct with probability exceeding $1 - 1/n^2$.
\pagebreak

% start of Q2
\begin{question}[]
\textbf{Rumour Spreading}
\end{question}
We will partition the experiment into following three stages. Let $X$ is the number of persons knowing the rumour at any time.
The stages are - 
\begin{enumerate}
    \item $X < c\log n$
    \item $c\log n < X < \frac{n}{2}$
    \item $\frac{n}{2} < X < n$
\end{enumerate}
Expected no. of days to spread the rumour is the sum of expected no. of days spent in each of these stages.

\textbf{(1)} Expected no. days spent in stage 1 - 

Let $p$ is the probability that no new person comes to know about the rumour at the end of some day. Let $k$ be the no. of persons knowing the rumour at the start of the day.

Thus,
$$p = \left(\frac{k}{n}\right)^k$$
For $1 \leq k < \frac{n}{2}$, $p \leq \left(\frac{1}{2}\right)^k \leq \frac{1}{2}$

So, expected no. of days for at least one person to know the rumour is $\frac{1}{1 - p} \leq 2$.

Thus the expected no. of days for $c\log n$ persons to know the rumour is less than or equal to $2c\log n$.
Expected no. of days spent in first stage, $$N_1 = 2c\log n$$

\textbf{(2)} Expected no. days spent in stage 2 - 



\textbf{(3)} Expected no. days spent in stage 3 - 

Let $k$ be the no. of persons knowing the rumour at the start of the day and let $r$ be the no. of persons not knowing the rumour at the start of the day ($r = n-k$).

Let $p_i$ be the probability that $i^{th}$ person does not know the rumour at the end of the day.
$$p_i = \left(\frac{n-1}{n}\right)^k$$
Since $k > \frac{n}{2}$,
$$p_i \leq \left(1 - \frac{1}{n}\right)^{\frac{n}{2}} \approx \frac{1}{\sqrt{e}}$$
Let $R_i$ be the random variable which takes value $1$ if the $i^{th}$ person does not know the rumour at the end of the day and $0$ otherwise.

Let $R$ be the no. of persons not knowing the rumour at the end of the day.

Now by linearity of expectation,
$$E[R] = \sum_{i=1}^r E[R_i]$$
$$E[R] = \sum_{i=1}^r p_i + 0(1- p_i)$$
$$E[R] \leq \frac{r}{\sqrt{e}}$$
Define a day to be good if the no. of persons not knowing the rumour reduces by more than $\frac{1}{\sqrt{2}}$ from the start of the day.

Thus the expected no. of good days required for spreading the rumour to $n$ people is $2\log n$.

Let $p$ be the probability that a day is bad,
$$p = P\left[R \geq \frac{r}{\sqrt{2}}\right]$$
Using Markov's inequality,
$$p \leq \frac{\sqrt{2}E[R]}{r}$$
$$p \leq \frac{\sqrt{2}}{\sqrt{e}} \leq \frac{7}{8}$$
and so probability of a day being good is at least $\frac{1}{8}$.

Thus the Expected no. of days spent in third stage, $$N_3 = 16\log n = q \log n$$

\pagebreak
% start of Q3
\begin{question}[]
\textbf{Approximate Ham-Sandwich Cut}

\end{question}


\end{document}

